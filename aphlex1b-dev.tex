% Initial System Characterization

\documentclass[9pt]{article} % use larger type; default would be 10pt

\usepackage[utf8]{inputenc} % set input encoding (not needed with XeLaTeX)
\usepackage{amsmath}
\usepackage{graphicx}
\usepackage{multicol}
\setlength{\columnsep}{0.5cm}
\usepackage[backend=biber]{biblatex}
\usepackage[margin=1.0in]{geometry}
\addbibresource{references.bib}

\numberwithin{equation}{section} % number within sections
% \usepackage[parfill]{parskip} % new line without indent

%%% Examples of Article customizations
% These packages are optional, depending whether you want the features they provide.
% See the LaTeX Companion or other references for full information.

%%% PAGE DIMENSIONS
\usepackage{changepage} % to adjust margins for a single paragraph
\usepackage{geometry} % to change the page dimensions
\geometry{a4paper} % or letterpaper (US) or a5paper or....
% \geometry{margin=2in} % for example, change the margins to 2 inches all round
% \geometry{landscape} % set up the page for landscape
%   read geometry.pdf for detailed page layout information

\usepackage{graphicx} % support the \includegraphics command and options

% \usepackage[parfill]{parskip} % Activate to begin paragraphs with an empty line rather than an indent

%%% PACKAGES
\usepackage{booktabs} % for much better looking tables
\usepackage{array} % for better arrays (eg matrices) in maths
\usepackage{paralist} % very flexible & customisable lists (eg. enumerate/itemize, etc.)
\usepackage{verbatim} % adds environment for commenting out blocks of text & for better verbatim
\usepackage{subfig} % make it possible to include more than one captioned figure/table in a single float
% These packages are all incorporated in the memoir class to one degree or another...

%%% HEADERS & FOOTERS
\usepackage{fancyhdr} % This should be set AFTER setting up the page geometry
\pagestyle{fancy} % options: empty , plain , fancy
\renewcommand{\headrulewidth}{0pt} % customise the layout...
\lhead{}\chead{}\rhead{}
\lfoot{}\cfoot{\thepage}\rfoot{}

%%% SECTION TITLE APPEARANCE
\usepackage{sectsty}
\allsectionsfont{\sffamily\mdseries\upshape} % (See the fntguide.pdf for font help)
% (This matches ConTeXt defaults)

%%% ToC (table of contents) APPEARANCE
\usepackage[nottoc,notlof,notlot]{tocbibind} % Put the bibliography in the ToC
\usepackage[titles,subfigure]{tocloft} % Alter the style of the Table of Contents
\renewcommand{\cftsecfont}{\rmfamily\mdseries\upshape}
\renewcommand{\cftsecpagefont}{\rmfamily\mdseries\upshape} % No bold!

%%% TITLE APPEARANCE (custom)
\usepackage[affil-it]{authblk} 
\usepackage{etoolbox}
\usepackage{lmodern}
%\usepackage{titling}
\makeatletter
\patchcmd{\@maketitle}{\LARGE \@title}{\fontsize{17}{20.2}\selectfont\@title}{}{}
\makeatother
\renewcommand\Authfont{\fontsize{12}{14.4}\selectfont}
\renewcommand\Affilfont{\fontsize{10}{12.8}\itshape}
%%% END Article customizations

%%% The "real" document content comes below...
\title{Development of a Cost-Effective 1.5kN Liquid-Fueled Rocket Propulsion System}
\author[1]{Jason Y. Chen \footnote{contact@projectcaelus.org,  jay.chen135@gmail.com}}
\affil[1]{Founder, Project Caelus 501(c)(3)}
\date{} % Activate to display a given date or no date (if empty), otherwise the current date is printed
\begin{document}
\maketitle
\vspace{-1cm}
\begin{center}
(Initial revision 06 August, 2019; received 02 February, 2020)
\end{center}

%%% ABSTRACT
\begin{adjustwidth}{40pt}{40pt}
\hspace{\parindent} This is an example of the abstract. Mit der CNN.com Europe Edition verfügt der TV-Sender CNN über eine umfassende Webseite, die im Minutentakt rund um die Uhr aktualisierte Weltnachrichten aus einer europäischen Perspektive auf den Bildschirm bringt. Wichtige, globale Ereignisse, geordnet in den Rubriken breaking news, current news headlines oder in depth, werden mit Hilfe von Video- und Audioclips, Bildern, Karten, Profilen, Zeitachsen und Tatsachenberichten fundiert dargestellt. Die Nachrichten stützen sich auf die Expertenanalysen der CNN Korrespondenten. Die Berichte verweisen auf andere relevante CNN Reportagen und zu Meldungen anderer Websites. Zusätzlich besteht die Möglichkeit, Videos anzufordern und mittels einer Suchfunktion Zugriff auf bereits gesendete Berichte und gesonderte Themenbereiche, die seit 1995 bearbeitet wurden, zu erhalten.

\end{adjustwidth}
\vspace{0.5cm}
\section{Nomenclature} \label{sec: nomenclature}
\vspace{0.1cm}
%\begin{center}
\begin{tabular}{lll}
\textbf{Symbols} \\ \\
$C_{v}$ & = \quad Valve flow coefficient & $$ \\
$f_{d}$ & = \quad Friction factor & $$ \\
$I_{sp} $ & = \quad Specific impulse & $s$ \\
$Q $ & = \quad Volumetric flow rate & $L/s$ \\
\end{tabular} 
\begin{tabular}{ll}
\textbf{Acronyms} \\ \\
$CEA $ & = \quad Chemical equilibrium w/ applications \\
$COTS $ & = \quad Commercial off-the-shelf \\
$GLOW $ & = \quad Gross lift-off weight \\
$SF $ & = \quad Safety factor
\end{tabular} \newline
%\end{center}

%\begin{multicols}{2}
\section{Initial System Characterization}
The following section outlines the characterization process used given certain mission constraints and objectives.
\subsection{Objectives}
The Callisto 1 system is set to the following objectives:
\begin{enumerate}
\item Reach an altitude of $1500 m$ ($\approx 5000 ft$).
\item A $GLOW$ of no more than $30 kg$ ($\approx 70 lbsm$).
\item A nominal main engine thrust of $1.5 kN$ ($\approx 350 lbsf$).
\item A chamber pressure in the range of around $1.5 MPa \text{to} 2.0 MPa$ ($\approx 218 psi \text{to} 300 psi$)
\item Consume a budget of no more than \$10,000 USD.
\item Utililize $95\%$ ethyl alcohol and nitrous oxide as the propellant combination.
\end{enumerate}
\subsection{Aphlex 1B}
The first step is to realize the theoretical maximum performance to be expected from our propellant combination.  95\% ethyl alcohol (ethanol) was chosen for its availability, low pricing, and a modest specific impulse. A 95\% dilution (by mass) with water was chosen as to lower the expected combustion chamber temperature while not sacrificing too much $I_{sp}$. Industrial nitrous oxide was chosen for its self-pressurizing characteristics, non-cryogenic nature as opposed to liquid oxygen, relative ease to obtain, and a modest $I_{sp}$ with ethanol.
Using $CEA$, an open-source thermodynamics library provided by NASA's Glenn Research Center, critical data relating propellant combustion characteristics could be obtained. The MATLAB plots below show the results for out analysis:

\subsection{Callisto 1}

\subsection{Plumbing System}
The following section describes the theoretical process used to dimensionalize the prelimiary plumbing framework in preparation for the first static cold flow test. By definition, these calculations are purely speculatory and are only used for the initial design process. The purpose of the cold flow test is to verify these parameters, and consequently adjust these parameters to better fit the system requirements.
\subsubsection{Assumptions} \label{sec:assumptions}

To simplify the rigorous analysis and optimization processes often associated with viscous pipe flow, a couple of assumptions are applied in the following section to both shorten the development timeline and to avoid unecessarily complex or expensive methods outside of the scope of a high school amateur rocketry program. They are as follows:
\begin{enumerate}
\item Flow is driven by both pressure and gravity.
\item Pipe is circular and is of constant cross-sectional area. \label{itm:constant-area}
\item No swirl, circumferential variation, or entrance effects.
\item No shaft-work or heat-transfer effects. \label{itm:heat-effects}
%\item A simplified steady-flow energy equation due to Assumption \ref{itm:heat-effects}.
\item Flow is fully developed (minimal boundary-layer effects).
\end{enumerate}
%\end{multicols}

\end{document}
