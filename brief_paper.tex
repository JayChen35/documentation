% !TEX TS-program = pdflatex
% !TEX encoding = UTF-8 Unicode

% This is a simple template for a LaTeX document using the "article" class.
% See "book", "report", "letter" for other types of document.

\documentclass[11pt]{article} % use larger type; default would be 10pt

\usepackage[utf8]{inputenc} % set input encoding (not needed with XeLaTeX)
\usepackage{amsmath}
\numberwithin{equation}{section} % number within sections
% \usepackage[parfill]{parskip} % new line without indent

%%% Examples of Article customizations
% These packages are optional, depending whether you want the features they provide.
% See the LaTeX Companion or other references for full information.

%%% PAGE DIMENSIONS
\usepackage{geometry} % to change the page dimensions
\geometry{a4paper} % or letterpaper (US) or a5paper or....
% \geometry{margin=2in} % for example, change the margins to 2 inches all round
% \geometry{landscape} % set up the page for landscape
%   read geometry.pdf for detailed page layout information

\usepackage{graphicx} % support the \includegraphics command and options

% \usepackage[parfill]{parskip} % Activate to begin paragraphs with an empty line rather than an indent

%%% PACKAGES
\usepackage{booktabs} % for much better looking tables
\usepackage{array} % for better arrays (eg matrices) in maths
\usepackage{paralist} % very flexible & customisable lists (eg. enumerate/itemize, etc.)
\usepackage{verbatim} % adds environment for commenting out blocks of text & for better verbatim
\usepackage{subfig} % make it possible to include more than one captioned figure/table in a single float
% These packages are all incorporated in the memoir class to one degree or another...

%%% HEADERS & FOOTERS
\usepackage{fancyhdr} % This should be set AFTER setting up the page geometry
\pagestyle{fancy} % options: empty , plain , fancy
\renewcommand{\headrulewidth}{0pt} % customise the layout...
\lhead{}\chead{}\rhead{}
\lfoot{}\cfoot{\thepage}\rfoot{}

%%% SECTION TITLE APPEARANCE
\usepackage{sectsty}
\allsectionsfont{\sffamily\mdseries\upshape} % (See the fntguide.pdf for font help)
% (This matches ConTeXt defaults)

%%% ToC (table of contents) APPEARANCE
\usepackage[nottoc,notlof,notlot]{tocbibind} % Put the bibliography in the ToC
\usepackage[titles,subfigure]{tocloft} % Alter the style of the Table of Contents
\renewcommand{\cftsecfont}{\rmfamily\mdseries\upshape}
\renewcommand{\cftsecpagefont}{\rmfamily\mdseries\upshape} % No bold!

%%% END Article customizations

%%% The "real" document content comes below...

\title{Development of a Cost-Effective 1kN Liquid-Fueled Rocket Propulsion System}
\author{%
	Jason Chen, Project Caelus 501(c)(3)  \\
	\large contact@projectcaelus.org}
%\date{} % Activate to display a given date or no date (if empty),
         % otherwise the current date is printed 

\begin{document}
\maketitle

\section{Pipe Dimension Calculation}

The following section describes the theoretical process used to dimensionalize the prelimiary plumbing framework in preparation for the first static cold flow test. By definition, these calculations are purely speculatory and are only used for the initial design process. The purpose of the cold flow test is to verify these parameters, and consequently adjust these parameters to better fit the system requirements.

\subsection{Assumptions} \label{sec:assumptions}

To simplify the rigorous analysis and optimization processes often associated with viscous pipe flow, a couple of assumptions are applied in the following section to both shorten the development timeline and to avoid unecessarily complex or expensive methods outside of the scope of a high school amateur rocketry program. They are as follows:
\begin{enumerate}
\item Flow is driven by both pressure and gravity.
\item Pipe is circular and is of constant cross-sectional area.
\item No swirl, circumferential variation, or entrance effects.
\item No shaft-work or heat-transfer effects. \label{itm:heat-effects}
\item A simplified steady-flow energy equation due to Assumption \ref{itm:heat-effects}.
\item Flow is fully developed (minimal boundary-layer effects).

\end{enumerate}


\subsection{Iterative Diameter Calculation}


Given the simplifying assumptions mentioned in Section \ref{sec:assumptions}, we can begin by introducing the dimensionless Darcy friction factor $f$ as a baseline relationship between roughness and pipe resistance:
\begin{equation} \label{eq:darcy-friction}
\frac{8 \tau_{w}}{\rho V^{2}} = f = F(Re_{d}, \frac{\epsilon}{d})
\end{equation}
where $\tau_{w}$ depicts the wall shear stress, $\rho$ is density, $V$ is volume, and $F$ represents a later-defined function between the Reynold's Number $Re_{d}$ and the average pipe roughness to diameter ratio $\frac{\epsilon}{d}$ (also known as relative roughness).

Using Equation \ref{eq:darcy-friction} and the wall shear stress equation (see References), we obtain the desired expression for finding pipe-head loss, also known as the Darcy-Weisbach equation. It is valid for duct flows of any cross section and any Reynold's Number
\begin{equation} \label{eq:darcy-weisbach}
h_{f} = f{\frac{L}{d}}{\frac{V^{2}}{2{g}}}
\end{equation}
where $h_{f}$ represents the head loss factor as a function of the dimensionless Darcy friction factor $f$, $L$ is the length of the pipe, and $g$ is equal to one standard Earth gravity.

\subsection{Units} \label{sec:units}
All units are implied with accordance to the Metric system (seconds, kilograms, Pascals, etc.), but are defined explicity below for ease-of-use:
\end{document}
