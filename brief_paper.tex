% !TEX TS-program = pdflatex
% !TEX encoding = UTF-8 Unicode

% This is a simple template for a LaTeX document using the "article" class.
% See "book", "report", "letter" for other types of document.

\documentclass[11pt]{article} % use larger type; default would be 10pt

\usepackage[utf8]{inputenc} % set input encoding (not needed with XeLaTeX)
\usepackage{amsmath}
\usepackage{graphicx}
\usepackage[backend=biber]{biblatex}
\addbibresource{references.bib}

\numberwithin{equation}{section} % number within sections
% \usepackage[parfill]{parskip} % new line without indent

%%% Examples of Article customizations
% These packages are optional, depending whether you want the features they provide.
% See the LaTeX Companion or other references for full information.

%%% PAGE DIMENSIONS
\usepackage{geometry} % to change the page dimensions
\geometry{a4paper} % or letterpaper (US) or a5paper or....
% \geometry{margin=2in} % for example, change the margins to 2 inches all round
% \geometry{landscape} % set up the page for landscape
%   read geometry.pdf for detailed page layout information

\usepackage{graphicx} % support the \includegraphics command and options

% \usepackage[parfill]{parskip} % Activate to begin paragraphs with an empty line rather than an indent

%%% PACKAGES
\usepackage{booktabs} % for much better looking tables
\usepackage{array} % for better arrays (eg matrices) in maths
\usepackage{paralist} % very flexible & customisable lists (eg. enumerate/itemize, etc.)
\usepackage{verbatim} % adds environment for commenting out blocks of text & for better verbatim
\usepackage{subfig} % make it possible to include more than one captioned figure/table in a single float
% These packages are all incorporated in the memoir class to one degree or another...

%%% HEADERS & FOOTERS
\usepackage{fancyhdr} % This should be set AFTER setting up the page geometry
\pagestyle{fancy} % options: empty , plain , fancy
\renewcommand{\headrulewidth}{0pt} % customise the layout...
\lhead{}\chead{}\rhead{}
\lfoot{}\cfoot{\thepage}\rfoot{}

%%% SECTION TITLE APPEARANCE
\usepackage{sectsty}
\allsectionsfont{\sffamily\mdseries\upshape} % (See the fntguide.pdf for font help)
% (This matches ConTeXt defaults)

%%% ToC (table of contents) APPEARANCE
\usepackage[nottoc,notlof,notlot]{tocbibind} % Put the bibliography in the ToC
\usepackage[titles,subfigure]{tocloft} % Alter the style of the Table of Contents
\renewcommand{\cftsecfont}{\rmfamily\mdseries\upshape}
\renewcommand{\cftsecpagefont}{\rmfamily\mdseries\upshape} % No bold!

%%% END Article customizations

%%% The "real" document content comes below...

\title{Development of a Cost-Effective 1kN Liquid-Fueled Rocket Propulsion System}
\author{%
	Jason Chen, Project Caelus 501(c)(3)  \\
	\large contact@projectcaelus.org}
%\date{} % Activate to display a given date or no date (if empty),
         % otherwise the current date is printed 

\begin{document}
\maketitle

\section{Pipe Dimension Calculation}

The following section describes the theoretical process used to dimensionalize the prelimiary plumbing framework in preparation for the first static cold flow test. By definition, these calculations are purely speculatory and are only used for the initial design process. The purpose of the cold flow test is to verify these parameters, and consequently adjust these parameters to better fit the system requirements.

\subsection{Assumptions} \label{sec:assumptions}

To simplify the rigorous analysis and optimization processes often associated with viscous pipe flow, a couple of assumptions are applied in the following section to both shorten the development timeline and to avoid unecessarily complex or expensive methods outside of the scope of a high school amateur rocketry program. They are as follows:
\begin{enumerate}
\item Flow is driven by both pressure and gravity.
\item Pipe is circular and is of constant cross-sectional area. \label{itm:constant-area}
\item No swirl, circumferential variation, or entrance effects.
\item No shaft-work or heat-transfer effects. \label{itm:heat-effects}
%\item A simplified steady-flow energy equation due to Assumption \ref{itm:heat-effects}.
\item Flow is fully developed (minimal boundary-layer effects).

\end{enumerate}
\subsection{The Reynolds Transport Theorem}
We can begin by introducing the concept of control-volume analysis \textemdash{} a powerful approach to solving fundamental fluid mechanics problems \textemdash{} in which a mathematical abstraction is employed to generate mathematical models of physical systems. That is, in an inertial frame of reference, the \textit{control volume} is a volume fixed in space or moving with constant velocity with the fluid flow. The surface enclosing the control volume is referred to as the \textit{control surface}. To convert a system analysis to a control-volume analysis, we must shift our perspective such that our mathematics apply to a specific region rather than to individual bodies. This process is called the \textit{Reynolds Transport Theorem}, and allows us to study certain properties (e.g. angular momentum $L$, enthalpy $h$, etc.) crossing the boundaries of the region. Essentially, we need to relate the time derivative of a system property of that property within a certain region.
\begin{figure}[h] 
\centering
\includegraphics[scale=0.5]{control_types}
\caption{Fixed, moving, and deformable control volumes: (a) fixed control volume for nozzle stress analysis; (b) control volume moving at ship speed for drag force analysis; (c) control volume deforming within cylinder for transient pressure variation analysis.}
\label{fig:control-types} % Always put labels after captions: https://tex.stackexchange.com/questions/32325/why-does-an-environments-label-have-to-appear-after-the-caption/32326#32326
\end{figure}

The desired conversion formula differs slightly according to whether the control volume is fixed, moving, or deformable, but for the purposes of our application, we'll be focusing on a fixed control volume. The fixed control volume in Figure \ref{fig:control-types}$a$ encloses a stationary region of interest to a nozzle designer. The control surface is an abstract concept that does not hinder the flow in any way. It slices through the jet leaving the nozzle, circles around through the surrounding (ambient) atmosphere, and slices through the flange bolts and the fluid within the nozzle \cite{fluid-mechanics}.

This particular control volume exposes the stresses in the flange bolts, which contribute to applied forces in the momentum analysis. In this sense, the control volume resembles the \textit{free-body} concept, which is often applied in solid-mechanics analysis \cite{fluid-mechanics}.
\begin{figure}[h] 
\centering
\includegraphics[scale=0.5]{control_volume}
\caption{An arbitrary fixed control volume with an arbitrary flow pattern.}
\label{fig:control-volume}
\end{figure}

Figure \ref{fig:control-volume} shows a generalized fixed control volume with an arbitrary flow pattern passing through. In general, each differential $dA$ of surface will have a different velocity vector $\vec{V}$ making a different angle $\theta$ with the local normal to $dA$. This gives us the general Reynolds transport theorem equation for an arbitrary fixed control volume:
\begin{equation} \label{eq:long-rtt}
\frac{d}{dt}(B_{syst}) = \frac{d}{dt} \left( \int_{CV} \beta \rho\, d\vartheta \right) + \int_{CS} \beta \rho\, cos\theta\, dA_{out} - \int_{CS} \beta \rho\, cos\theta\, dA_{in}
\end{equation}
Asserting the property $B$ to be either mass, momentum, angular momentum, or energy, all basic laws can be rewritten in control-volume form. Note that all three control volume integrals are a function of the intensive propery $\beta$, and since the control volume is fixed in space, the elemental volumes $d\vartheta$ do not vary with time, and thus the time derivative of the volume integral disappears unless $\beta$ or $\rho$ varies with time (unsteady flow) \cite{fluid-mechanics}.

However, realizing that if $\vec{n}$ is defined as the \textit{outward} normal unit vector everywhere on the control surface, then $\vec{V} \cdot \vec{n} = V_{n}$ for outflow and $\vec{V} \cdot \vec{n} = -V_{n}$ for inflow. Therefore, the flux terms can be combined and Equation \ref{eq:long-rtt} can be rewritten as
\begin{equation} \label{eq:short-rtt}
\frac{d}{dt}(B_{syst}) = \frac{d}{dt} \left( \int_{CV} \beta \rho\, d\vartheta \right) + \int_{CV} \beta \rho(\vec{V} \cdot \vec{n})\, dA
\end{equation}
We are now able to apply the principles of conservation of mass to the Reynolds transport theorem by setting the arbitrary variable $B$ equal to mass, and $\beta = dm/dm =1$. Equation \ref{eq:short-rtt} becomes
\begin{equation}
\left( \frac{dm}{dt} \right)_{syst} = 0 = \frac{d}{dt} \left( \int_{CV} \rho\, d\vartheta \right) + \int_{CS} \rho(\vec{V_{r}} \cdot \vec{n})\, dA
\end{equation}
This is the integral mass-conservation law for a deformable control volume. Adapting this for a fixed control volume and simplying gives
\begin{equation} \label{eq:mass-conservation-law}
\int_{CV} \frac{\partial \rho}{\partial t}\, d\vartheta + \int_{CS} \rho(\vec{V} \cdot \vec{n})\, dA = 0
\end{equation}
Finally, knowing that we are dealing with steady, incompressible flow through the control volume ($\partial \rho/ \partial t = 0$), Equation \ref{eq:mass-conservation-law} reduces to
\begin{equation} \label{eq:continuity-relation}
\int_{CS} \rho(\vec{V} \cdot \vec{n})\, dA = 0
\end{equation}
This equation explicitly states that in steady flow, the mass flows entering and exitting the control volume are exactly equal, while neglecting sources or sinks of mass which might be embedded in the control volume.

\subsection{Viscous Flow in a Circular Pipe}

Now that a theoretical background has been established, we can begin to discuss the specific case of solving for flow rates, pressure drops, head losses, and diameter sizings via incompressible, viscous flows in circular pipes. Given the simplifying assumptions mentioned in Section \ref{sec:assumptions}, we can start simply with the classic case of Bernoulli's Equation:
\begin{equation} \label{eq:bernoulli}
p + \frac{1}{2} \rho V^{2} + \rho g z = constant
\end{equation}
where $p$ is the pressure, $\rho$ is density, $V$ is velocity, $z$ is elevation, and $g$ represents gravitational acceleration. It gives insight into the balance between pressure, velocity, and elevation by assuming that the two points in question lie on a streamline, the fluid is incompressible, the flow is steady and inviscid, and there is no friction. Although useful for some applications, it is not adequate for designing robust piping systems which will in actuality encounter such effects.

The addition of a \textit{head loss} term, denoted as $h_{f}$, is the first step in introducing a viscous term into an otherwise inviscid equation. Head loss can simply be thought of as an additional pressure loss in the system due to viscous effects, a sudden expansion/contraction, and/or obstructions to its path such as pipe elbows, bends, valves, etc. This can be done by first recalling the continuity (conservation of mass) relation (Equation \ref{eq:continuity-relation}). Realizing that the pipe is of constant area and assuming only a number of one-dimensional inlets and outlets, Equation \ref{eq:continuity-relation} reduces to
\begin{equation} \label{eq:equal-q}
Q_{1} = Q_{2} = constant
\end{equation}
or
\begin{equation} \label{eq:equal-v}
V_{1} = \frac{Q_{1}}{A_{1}} = V_{2} = \frac{Q_{2}}{A_{2}}
\end{equation}
Bernoulli's Equation (Equation \ref{eq:bernoulli}, in combination with the steady-flow energy equation from \cite{fluid-mechanics}) can thus be reduced and rearranged into
\begin{equation} \label{eq:steady-flow-energy}
\frac{p_{1}}{\rho} + \frac{1}{2} \alpha_{1} V^{2}_{1} + gz_{1} = \frac{p_{2}}{\rho} + \frac{1}{2} \alpha_{2} V^{2}_{2} + gz_{2} + gh_{f}
\end{equation}
due to Assumption \ref{itm:heat-effects}, where $\alpha$ denotes the kinetic-energy correlation factor (energy coefficient). $\alpha$ is 1 when the flow is fully developed and is less than 1 if the flow is underdeveloped. Note that this equation is also very similar to the result of dividing Equation \ref{eq:bernoulli} by the specific weight of the fluid, $\rho \cdot g$.
\begin{figure}[h]
\centering
\includegraphics[scale=0.45]{inclined_pipe}
\caption{Control volume of a steady, fully-developed flow between two sections of an inclined pipe.}
\label{fig:inclined-pipe}
\end{figure}
For fully-developed flow, the velocity profile shape is the same at sections 1 and 2, as seen in Figure \ref{fig:inclined-pipe}. Thus the kinetic-energy correlation factor $\alpha_{1} = \alpha_{2}$, and since $V_{1} = V_{2}$ from Equation \ref{eq:equal-v}, Equation \ref{eq:steady-flow-energy} reduces to a simple expression for the friction-head loss $h_{f}$ \cite{fluid-mechanics}
\begin{equation}
h_{f} = (z_{1} - z_{2}) + \left( \frac{p_{1}}{\rho g} - \frac{p_{2}}{\rho g} \right) = \Delta z + \frac{\Delta p}{\rho g}
\end{equation}
Applying the one-dimensional momentum relation
\begin{equation}
\sum \vec{F} = \frac{d}{dt} \left( \int_{CV} \vec{V} \rho\, d\vartheta \right) + \sum (\dot{m_{i}} \vec{V}_{i})_{out} - \sum (\dot{m_{i}} \vec{V}_{i})_{in}
\end{equation}
to the control volume in Figure \ref{fig:inclined-pipe} while accounting for applied $x$-directed forces due to pressure, gravity, and shear:
\begin{equation}
\sum F_{x} = \Delta p (\pi R^{2}) + \rho g (\pi R^{2}) L\, sin \phi - \tau_{w}(2 \pi R)L = \dot{m} (V_{2} - V_{1}) = 0
\end{equation}
Rearranging shows that head loss $h_{f}$ is also related to wall shear stress
\begin{equation} \label{eq:wall-shear-stress}
\Delta z + \frac{\Delta p}{\rho g} = h_{f} = \frac{2 \tau_{w}}{\rho g} \frac{L}{R} = \frac{4 \tau{w}}{\rho g} \frac{L}{d}
\end{equation}
where we have substituted $\Delta z = \Delta L\, sin \phi$ from Figure \ref{fig:inclined-pipe}. Note that regardless the value of $\phi$ (whether the pipe is horizontal or tilted), the head loss is proportional to the wall shear stress.

Finally, to correlate the head loss term into a useful form for solving pipe flow problems, we have the famous \textit{Darcy-Weisbach equation}, which is a proposed correlation valid for duct flow of any cross section and any Reynolds Number:
\begin{equation} \label{eq:darcy-weisbach}
h_{f} = f{\frac{L}{d}}{\frac{V^{2}}{2{g}}}
\end{equation}
The dimensionless parameter $f$ is the \textit{Darcy friction factor}, which establishes a relationship between roughness and pipe resistance. $\epsilon$ is the wall roughness height, which is significant only in turbulent pipe flow. It has been shown that $\epsilon$ has orders of magnitude \textit{less of an effect} on laminar pipe flow. By equating Equations \ref{eq:wall-shear-stress} and \ref{eq:darcy-weisbach} we find an alternative form of the friction factor:
\begin{equation} \label{eq:darcy-friction}
\frac{8 \tau_{w}}{\rho V^{2}} = f = F(Re_{d}, \frac{\epsilon}{d})
\end{equation}
where $F$ represents some relationship between the Reynolds Number $Re_{d}$ and the average pipe roughness to diameter ratio $\epsilon / d$ (also known as relative roughness).

\subsection{Laminar Fully Developed Pipe Flow}
The following section briefly outlines the derivations and methods necessary to solve specifically a laminar flow problem. 

\begin{table}[!htb]
\centering
\begin{tabular}{ |p{4cm}||p{4cm}|p{2cm}|p{2cm}|  }
\hline
\multicolumn{4}{|c|}{Design Parameters} \\
\hline
Name & Value & Unit & Uncertainty \\
\hline
Propellant  &  Isopropyl alcohol ($95\%$)   &  N/A  &  N/A \\
$\mu$, Dynamic viscosity  &  0.00196  &  $kg/m \cdot s$  &  $\pm 0.1 \%$ \\
$\rho$, Density  &  786  &  $kg/m^{3}$  &  $\pm 1 \%$ \\
$L$, Pipe length  &  8.0  &  $m$  &  $\pm 0.1 \%$  \\
$d$, Pipe diameter  &  12.7  &  $mm$  &   $\pm 0.1 \%$ \\
$\dot{m}$, Mass flow rate  &  0.9133  &  $kg/s$  &   $\pm 0.1 \%$ \\
Pipe material  &  Hard nylon  &  N/A  &  N/A \\
$\epsilon$, Roughness  &  1.5 to 40.0  &  $\mu m$  &  $\pm 50 \%$ \\
\hline
\end{tabular}
\caption{Design parameters and fluid properties required for pipe calculations.}
\label{table:parameters}
\end{table}
However, by just doing a few simple calculations using values from Table \ref{table:parameters}, we quickly realize our flow is most likely turbulent:
\begin{equation}
Q = \frac{\dot{m}}{\rho} = \frac{0.9133\, kg/s}{786\, kg/m^{3}} = 1.162 \times 10^{-3}\, m^{3}/s
\end{equation}
\begin{equation}
V = \frac{Q}{\pi R^{2}} = \frac{(1.162 \times 10^{-3}\, m^{3}/s)}{\pi (0.0127 / 2\, m)^{2}} = 9.173\, m/s
\end{equation}
\begin{equation}
\nu = \frac{\mu}{\rho} = \frac{0.00196\, kg/m \cdot s}{786\, kg/m^{3}} = 2.494 \times 10^{-6}\, m^{2}/s
\end{equation}
\begin{equation}
Re_{d} = \frac{V d}{\nu} = \frac{(9.173\, m/s)(0.0127\, m)}{2.494 \times 10^{-6}\, m^{2}/s} = 46711
\end{equation}
since $Re_{d} >> 2300$. The purpose of this section, therefore, is to introduce and highlight key differences when dealing with laminar versus turbulent flow when designing a pipe system. The benefits and drawbacks of either flow type will become apparent, and this will hopefully provide the reader with stronger intuition on the design and manufacturing tradeoffs associated with each flow type. If the reader is only interested in equations relevant to our specific calculations, the next section covers the solutions to turbulent fully developed pipe flow.

\begin{figure}[!htb] %[!htb] puts more than one figure on the same page
\centering
\includegraphics[scale=0.46]{developing_velocity_profile}
\caption{Developing velocity profiles and pressure changes in the entrance of a duct.}
\label{fig:developing-velocity-profile}
\end{figure}
\begin{figure}[!htb]
\centering
\includegraphics[scale=0.35]{comparing_velocity_profiles}
\caption{Comparison of laminar and turbulent pipe flow velocity profiles for the same volume flow: $(a)$ laminar flow; $(b)$ turbulent flow.}
\label{fig:comparing-velocity-profiles}
\end{figure}

Considering a fully-developed \textit{Hagen-Poiseuille flow} (which represents the laminar distribution given in Figure \ref{fig:comparing-velocity-profiles}) in a round pipe of diameter $d$ and radius $R$, complete analytical solutions can be easily derived from formulas given in \textcite{fluid-mechanics}'s Section 4.10. A compilation of those formulas is given below:
\begin{equation} \label{eq:poiseuille-v}
V = \frac{Q}{A} = \frac{u_{max}}{2} = \left( \frac{\Delta p + \rho g \Delta z}{L} \right) \frac{R^{2}}{8 \mu}
\end{equation}
\begin{equation}
Q = \int u\, dA = \pi R^{2} V = \left( \frac{\Delta p + \rho g \Delta z}{L} \right) \frac{\pi R^{4}}{8 \mu}
\end{equation}
\begin{equation} \label{eq:poiseuille-tw}
\tau_{w} = \frac{4 \mu V}{R} = \frac{8 \mu V}{d} = \frac{R}{2} \left( \frac{\Delta p + \rho g \Delta z}{L} \right)
\end{equation}
\begin{equation}
h_{f} = \frac{32 \mu L V}{\rho g d^{2}} = \frac{128 \mu L Q}{\pi \rho g d^{4}}
\end{equation}


To briefly explain the laminar distribution shown in Figure \ref{fig:comparing-velocity-profiles}, we start by evaluating off the assumption of the no-slip condition at the wall (i.e. $u=0$ and $r=R$ in Figure \ref{fig:inclined-pipe}) to obtain the velocity solution for laminar fully-developed pipe flow
\begin{equation} \label{eq:laminar-pipe-flow}
u = \frac{1}{4 u} \big[ -\frac{d}{dx} (p + \rho g z) \big] (R^{2} - r^{2})
\end{equation}
The detailed derivations of Equation \ref{eq:laminar-pipe-flow} are not shown here, but it is notable that Equation \ref{eq:laminar-pipe-flow} is based off the \textit{hydraulic grade line} (HGL) equations while knowing that for laminar flow $\tau = \mu du/dr$ \cite{fluid-mechanics}.

Therefore, the laminar flow profile is a paraboloid falling to zero at the wall and reaching a maximum at the axis
\begin{equation}
u_{max} = \frac{R^{2}}{4 \mu} \big[  -\frac{d}{dx} (p + \rho g z) \big]
\end{equation}
The paraboloid profile, characteristic of Poiseuille flow, has an average velocity $V$ which is one-half of the maximum velocity (also shown in Equation \ref{eq:poiseuille-v}). The quantity $\Delta p$ is the pressure \textit{drop} in a pipe of length $L$. These formulas are valid whenever the pipe Reynolds number, $Re_{d} = \rho V d/\mu$, is less than about 2300 (this is the widely accepted threshold between laminar and turbulent flow). Note that $\tau_{w}$ is proportional to $V$ (see Figure \ref{fig:developing-velocity-profile}, Equation \ref{eq:poiseuille-tw}) and is independent of density because the fluid acceleration is zero. Neither of these is true in turbulent flow \cite{fluid-mechanics}.

If we know the wall shear stress, the Poiseuille flow friction factor is
\begin{equation} \label{eq:friction-factor}
f_{lam} = \frac{8 \tau_{w, lam}}{\rho V^{2}} = \frac{8(8 \mu V/d)}{\rho V^{2}} = \frac{64}{\rho V d/\mu} = \frac{64}{Re_{d}}
\end{equation}	
A clear relation between two terms in Equation \ref{eq:friction-factor} arises \textemdash{} the pipe friction factor decreases inversely with Reynolds Number.
\subsection{Turbulent Pipe Flow}


\subsection{Example Calculation}
\subsubsection{Determining Head Loss and Pressure Drop Using the Moody Chart}

\subsection{Units and Symbols} \label{sec:units}
All units are implied with accordance to the Metric system (seconds, kilograms, Pascals, etc.), but are defined explicity along with common Greek symbols below for ease-of-use:
\begin{itemize}%[\label{}]
\item $Q$ = volumetric flow rate, $m^{3}/s$
\item $u$ = velocity, $m/s$ 
\item $u(r), V$ = local mean velocity, $m/s$ 
\item $u_{max}$ = local maximum velocity, $m/s$ 
\item $\mu$ = dynamic viscosity, $\frac{kg}{m \cdot s}$ or $Pa \cdot s$
\item $\nu$ = kinematic viscosity, $m^{2}/s$
\item $R$ = radius, $m$
\item $Re_{d}$ = Reynolds Number, dimensionless%. Laminar flow: $Re_{d} < 2300$ Turbulent flow: $2300 < Re_{d}$

\end{itemize}
\printbibliography
\end{document}
